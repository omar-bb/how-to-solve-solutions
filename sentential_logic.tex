\documentclass{article} % Document class

\usepackage[utf8]{inputenc} % Input encoding
\usepackage[letterpaper, margin=1in]{geometry} % Page layout
\usepackage{amsmath, amssymb} % Math packages
\usepackage{enumitem} % Package for customizing lists

% Customization of enumeration environment
\setlist[enumerate]{label=\arabic*., leftmargin=*, topsep=2pt, partopsep=2pt, parsep=2pt, itemsep=2pt}

% Header customization
\usepackage{fancyhdr}
\pagestyle{fancy}
\lhead{} % Empty left header
\rhead{Date: \today} % Right header content (change "\today" to desired date format)

\begin{document}
    \section*{Introduction} % Exercise set title (unnumbered)
    
    \subsection*{Exercises} % Section for exercises (unnumbered)
    
    \begin{enumerate}
        \item \textbf{Exercise 1:}
        \begin{itemize}
            \item[(a)] Let \(P\) stand for "We'll have a reading assignment", \(Q\) for "We'll have a homework problem" and \(R\) for "We'll have a test", then \[(P \vee Q) \wedge \neg (Q \wedge R)\].
            \item[(b)] Let \(P\) stand for "You will go skiing" and \(Q\) for "There will be snow", then \[\neg P \vee (P \wedge \neg Q)\]
            \item[(c)] \( \neg [(\sqrt{7} < 2) \wedge (\sqrt{7} - 2)] \)
        \end{itemize} 
    \end{enumerate}
\end{document}
    