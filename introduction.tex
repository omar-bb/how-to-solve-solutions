\documentclass{article} % Document class

\usepackage[utf8]{inputenc} % Input encoding
\usepackage[letterpaper, margin=1in]{geometry} % Page layout
\usepackage{amsmath, amssymb} % Math packages
\usepackage{enumitem} % Package for customizing lists

% Customization of enumeration environment
\setlist[enumerate]{label=\arabic*., leftmargin=*, topsep=2pt, partopsep=2pt, parsep=2pt, itemsep=2pt}

% Header customization
\usepackage{fancyhdr}
\pagestyle{fancy}
\lhead{} % Empty left header
\rhead{Date: \today} % Right header content (change "\today" to desired date format)

\begin{document}
    \section*{Introduction} % Exercise set title (unnumbered)
    
    \subsection*{Exercises} % Section for exercises (unnumbered)
    
    \begin{enumerate}
        \item \textbf{Exercise 1:}
        \begin{itemize}
            \item[(a)] We have \(2^{15} - 1 = 32767 = 1057 \cdot 31 = 4681 \cdot 7\).
            \item[(b)] We know that \(32767 = 1057 \cdot 31\), we have \(2^{32767} - 1\) divisible by \(2^{31} - 1\) (since \(2^{32767} - 1 = (2^{31} - 1)(1 + 2^{31} + 2^{2 \cdot 31} + \dots + 2^{1056 \cdot 31})\)) and \(1 < 2^{31} - 1 < 2^{32767} - 1\).
        \end{itemize}
        
        \item \textbf{Exercise 2:}
        \begin{table}[h]
            \begin{tabular}{llllll}
                \hline
                \textbf{\(n\)} & \textbf{is \(n\) prime?} & \textbf{\(3^{n} - 1\)} & \textbf{is \(3^{n} - 1\) prime?} & \textbf{\(3^{n} - 2^{n}\)} & \textbf{is \(3^{n} - 2^{n}\) prime?} \\ \hline
                2            & yes                    & 8                    & no                             & 5                        & yes                                \\
                3            & yes                    & 26                   & no                             & 19                       & yes                                \\
                4            & no                     & 80                   & no                             & 65                       & no                                 \\
                5            & yes                    & 242                  & no                             & 211                      & yes                                \\
                6            & no                     & 720                  & no                             & 665                      & no                                 \\
                7            & yes                    & 2186                 & no                             & 2059                     & no?                                \\ \hline
            \end{tabular}
        \end{table}

        \textbf{Conjecture 1.} Suppose \(n\) is an integer, if \(n > 0\) then \(3^{n} - 1\) is not prime.

        \item \textbf{Exercice 3:}
        \begin{itemize}
            \item[(a)] Let \(m = 2 \cdot  3 \cdot 5 \cdot 7 + 1 = 211\), by theorem 3 \(m\) is prime.
            \item[(b)] Let \(m = 2 \cdot 5 \cdot 11 + 1 = 111\) is not prime since \(111 = 3 \cdot 37\), thus it yields two primes.
        \end{itemize}

        \item \textbf{Exercice 4:}
        Let \(x = (5 + 1)! + 2 = 6! + 2\), Hence by theorem 4, we gave \(722, 723, 724, 725, 726\) are not primes.
        
        \item \textbf{Exercice 5:}
        We have \(2^{5} - 1 = 31\), Hence according to Euclid \(2^{5 - 1}(2^{5} - 1) = 496\) is perfect.

        \item \textbf{Exercice 6:}
        Let's suppose that there exist such "triplet primes": \((n, n + 2, n + 4)\) where \(n \geq 5\), \(n\) could be written as follows, either \(n = 3k + 1\) or \(n = 3k + 2\). If \(n = 3k + 1\) then \(n + 2 = 3k + 3 = 3(k + 1)\) which is divisible by 3, thus contradicting the fact that \(n + 2\) is prime and our supposition. If \(n = 3k + 2\) then \(n + 4 = 3k + 6 = 3(k + 2)\) which is divisible by 3, thus contradicting the fact that \(n + 4\) is prime. Hence in all cases of \(n\) we end up with a contradiction, we can then conclude that there are no "triplet primes" for all \(n \geq 5\).


        % Add more exercises as needed
    \end{enumerate}
    
\end{document}