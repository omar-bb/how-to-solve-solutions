\documentclass{article} % Document class

\usepackage[utf8]{inputenc} % Input encoding
\usepackage[letterpaper, margin=1in]{geometry} % Page layout
\usepackage{amsmath, amssymb} % Math packages
\usepackage{enumitem} % Package for customizing lists

% Customization of enumeration environment
\setlist[enumerate]{label=\arabic*., leftmargin=*, topsep=2pt, partopsep=2pt, parsep=2pt, itemsep=2pt}

% Header customization
\usepackage{fancyhdr}
\pagestyle{fancy}
\lhead{} % Empty left header
\rhead{Date: \today} % Right header content (change "\today" to desired date format)

\begin{document}
    \section*{Introduction} % Exercise set title (unnumbered)
    
    \subsection*{Exercises} % Section for exercises (unnumbered)
    
    \begin{enumerate}
        \item \textbf{Exercise 1:}
        \begin{itemize}
            \item[(a)] Let \(P\) stand for "We'll have a reading assignment", \(Q\) for "We'll have a homework problem" and \(R\) for "We'll have a test", then \[(P \vee Q) \wedge \neg (Q \wedge R)\].
            \item[(b)] Let \(P\) stand for "You will go skiing" and \(Q\) for "There will be snow", then \[\neg P \vee (P \wedge \neg Q)\]
            \item[(c)] \( \neg [(\sqrt{7} < 2) \wedge (\sqrt{7} - 2)] \)
        \end{itemize} 

        \item \textbf{Exercise 2:}
        \begin{itemize}
            \item[(a)] Let \(J\) stand for "John is telling the truth" and \(B\) for "Bill is telling the truth", then \[(J \wedge B) \vee (\neg J \wedge \neg B)\]
            \item[(b)] Let \(F\) stand for "I'll have fish", \(C\) for "I'll have chicken" and \(M\) for "I'll have mashed potatoes", then \[(F \vee C) \wedge \neg (F \wedge M)\]
            \item[(c)] Let \(X\) stand for "3 is a common divisor of 6", \(Y\) for "3 is a common divisor of 9" and \(Z\) for "3 is a common divisor of 15", then \[X \wedge Y \wedge Z\]
        \end{itemize}

        \item \textbf{Exercise 3:}
        Let us define some statements for the rest of the exercise. Let \(A\) stand for "Alice is in the room" and \(B\) for "Bob is in the room",
        \begin{itemize}
            \item[(a)] We have \(A \wedge B = \) "Alice and Bob are both in the room", then \[\neg (A \wedge B)\]
            \item[(b)] \(\neg A \wedge \neg B\)
            \item[(c)] \(\neg A \vee \neg B\)
            \item[(d)] \(\neg (A \vee B)\)
        \end{itemize}

        \item \textbf{Exercice 4:}
        \begin{itemize}
            \item[(a)] is well formed.
            \item[(b)] is not well formed.
            \item[(c)] is well formed.
            \item[(d)] is not well formed.
        \end{itemize}

        \item \textbf{Exercise 5:}
        \begin{itemize}
            \item[(a)] I won't buy both the pants without the shirt.
            \item[(b)] I won't buy the pants and I won't buy the shirt / I will neither buy the pants nor the shirt.
            \item[(c)] Either I won't buy the pants or won't buy the shirt.
        \end{itemize}

        \item \textbf{Exercise 6:}
        \begin{itemize}
            \item[(a)] Either Steve is happy or George is happy, but either of them is not.
        \end{itemize}
    \end{enumerate}
\end{document}
    