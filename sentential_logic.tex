\documentclass{article} % Document class

\usepackage[utf8]{inputenc} % Input encoding
\usepackage[letterpaper, margin=1in]{geometry} % Page layout
\usepackage{amsmath, amssymb} % Math packages
\usepackage{enumitem} % Package for customizing lists
\usepackage{booktabs}

% Customization of enumeration environment
\setlist[enumerate]{label=\arabic*., leftmargin=*, topsep=2pt, partopsep=2pt, parsep=2pt, itemsep=2pt}

% Header customization
\usepackage{fancyhdr}
\pagestyle{fancy}
\lhead{} % Empty left header
\rhead{Date: \today} % Right header content (change "\today" to desired date format)

\setlength{\parindent}{0pt}

\begin{document}
    \section{Deductive Reasoning and Logical Connectives} % Exercise set title (unnumbered)
    
    \subsection*{Exercises} % Section for exercises (unnumbered)
    
    \begin{enumerate}
        \item \textbf{Exercise 1:}
        \begin{itemize}
            \item[(a)] Let \(P\) stand for "We'll have a reading assignment", \(Q\) for "We'll have a homework problem" and \(R\) for "We'll have a test", then \[(P \vee Q) \wedge \neg (Q \wedge R)\].
            \item[(b)] Let \(P\) stand for "You will go skiing" and \(Q\) for "There will be snow", then \[\neg P \vee (P \wedge \neg Q)\]
            \item[(c)] \( \neg [(\sqrt{7} < 2) \wedge (\sqrt{7} - 2)] \)
        \end{itemize} 

        \item \textbf{Exercise 2:}
        \begin{itemize}
            \item[(a)] Let \(J\) stand for "John is telling the truth" and \(B\) for "Bill is telling the truth", then \[(J \wedge B) \vee (\neg J \wedge \neg B)\]
            \item[(b)] Let \(F\) stand for "I'll have fish", \(C\) for "I'll have chicken" and \(M\) for "I'll have mashed potatoes", then \[(F \vee C) \wedge \neg (F \wedge M)\]
            \item[(c)] Let \(X\) stand for "3 is a common divisor of 6", \(Y\) for "3 is a common divisor of 9" and \(Z\) for "3 is a common divisor of 15", then \[X \wedge Y \wedge Z\]
        \end{itemize}

        \item \textbf{Exercise 3:}
        Let us define some statements for the rest of the exercise. Let \(A\) stand for "Alice is in the room" and \(B\) for "Bob is in the room",
        \begin{itemize}
            \item[(a)] We have \(A \wedge B = \) "Alice and Bob are both in the room", then \[\neg (A \wedge B)\]
            \item[(b)] \(\neg A \wedge \neg B\)
            \item[(c)] \(\neg A \vee \neg B\)
            \item[(d)] \(\neg (A \vee B)\)
        \end{itemize}

        \item \textbf{Exercice 4:}
        \begin{itemize}
            \item[(a)] is well formed.
            \item[(b)] is not well formed.
            \item[(c)] is well formed.
            \item[(d)] is not well formed.
        \end{itemize}

        \item \textbf{Exercise 5:}
        \begin{itemize}
            \item[(a)] I won't buy both the pants without the shirt.
            \item[(b)] I won't buy the pants and I won't buy the shirt / I will neither buy the pants nor the shirt.
            \item[(c)] Either I won't buy the pants or won't buy the shirt.
        \end{itemize}

        \item \textbf{Exercise 6:}
        \begin{itemize}
            \item[(a)] Either Steve is happy or George is happy, but not both.
            \item[(b)] Either George is not happy or Steve is happy or George is happy and Steve is not. \\
            \dots
        \end{itemize}
        
        \item \textbf{Exercise 7:}
        \begin{itemize}
            \item[(a)] \textbf{The Premises:}
            \begin{align*}
                &\text{Jane and Pete won't both win the math prize.} \\
                &\text{Pete will win either the math prize or the chemistry prize.} \\
                &\text{Jane will win the math prize.}
            \end{align*}
            \textbf{The Conclusion:} Pete will win the chemistry prize. \\
            Let \(J\) stand for "Jane will win the math prize", \(P\) for "Pete will win the math prize", and \(C\) for "Pete will win the chemistry prize". \\
            \textbf{Logical Forms:}
            \begin{align*}
                &\text{Premise 1:} && \neg (J \wedge P) \\
                &\text{Premise 2:} && P \vee C \\
                &\text{Premise 3:} && J \\
                &\text{Conclusion:} && C 
            \end{align*} \\
            \textbf{Reasoning is valid:} The reasoning is valid since by the third premise Jane won the math prize, by the first premise Pete didn't win it and by the second premise Pete will either win the math or chemistry prize and since he didn't win the math prize, thus he won the chemistry prize.

            \item[(b)] \textbf{The Premises:}
            \begin{align*}
                &\text{The main course will be either beef or fish.} \\
                &\text{The vegetable will be either peas or corn.} \\
                &\text{We will not have both fish as a main course and corn as a vegetable.}
            \end{align*}
            \textbf{The Conclusion:} We will not have both beef as a main course and peas as a vegetable. \\
            Let \(B\) stand for "The main course will be beef", \(F\) for "The main course will be fish", \(P\) for "The vegetable will be peas" and \(C\) for "The vegetable will be corn" \\
            \textbf{Logical Forms:}
            \begin{align*}
                &\text{Premise 1:} && B \vee F \\
                &\text{Premise 2:} && P \vee C \\
                &\text{Premise 3:} && \neg (F \wedge C) \\
                &\text{Conclusion:} && \neg (B \wedge P) 
            \end{align*} \\
            \textbf{Reasoning is valid:} The reasoning isn't valid since you can't have both fish and corn, and you can either have beef or fish for the main course and either corn or peas as a vegetable so you can end up with beef and peas 

            \item[(b)] \textbf{The Premises:}
            \begin{align*}
                &\text{The main course will be either beef or fish.} \\
                &\text{The vegetable will be either peas or corn.} \\
                &\text{We will not have both fish as a main course and corn as a vegetable.}
            \end{align*}
            \textbf{The Conclusion:} We will not have both beef as a main course and peas as a vegetable. \\
            Let \(B\) stand for "The main course will be beef", \(F\) for "The main course will be fish", \(P\) for "The vegetable will be peas" and \(C\) for "The vegetable will be corn" \\
            \textbf{Logical Forms:}
            \begin{align*}
                &\text{Premise 1:} && B \vee F \\
                &\text{Premise 2:} && P \vee C \\
                &\text{Premise 3:} && \neg (F \wedge C) \\
                &\text{Conclusion:} && \neg (B \wedge P) 
            \end{align*} \\
            \textbf{Reasoning is valid:} The reasoning isn't valid since you can't have both fish and corn, and you can either have beef or fish for the main course and either corn or peas as a vegetable so you can end up with beef and peas 

            \item[(c)] \textbf{The Premises:}
            \begin{align*}
                &\text{Either John or Bill is telling the truth.} \\
                &\text{Either Sam or Bill is lying.} \\
            \end{align*}
            \textbf{The Conclusion:} Either John is telling the truth or Sam is lying. \\
            Let \(J\) stand for "John is telling the truth", \(B\) for "Bill is telling the truth" and \(S\) for "Sam is telling the truth" \\
            \textbf{Logical Forms:}
            \begin{align*}
                &\text{Premise 1:} && J \vee B \\
                &\text{Premise 2:} && \neg S \vee \neg B \\
                &\text{Conclusion:} && J \vee \neg S 
            \end{align*} \\
            \textbf{Reasoning is valid:} The reasoning seems valid since the only one who can affect the outcome of the premises combined is Bill since, if Bill is lying then John and Sam are telling the truth and if Bill is telling the truth then John and Sam are lying, These are the two possible outcomes, Hence the conclusion is right since Either John is telling the truth or Sam is lying but you can't have both

            \item[(d)] \textbf{The Premises:}
            \begin{align*}
                &\text{Either sales will go up and the boss will be happy, or expenses will go up and the boss won't be happy.}
            \end{align*}
            \textbf{The Conclusion:} Sales and expenses will not both go up. \\
            Let \(S\) stand for "The sales will go up", \(B\) for "The boss will be happy" and \(E\) for "The expenses will go up". \\
            \textbf{Logical Forms:}
            \begin{align*}
                &\text{Premise 1:} && (S \wedge B) \vee (E \wedge \neg B) \\
                &\text{Conclusion:} && \neg (S \wedge E) 
            \end{align*} \\
            \textbf{Reasoning is valid:} https://math.stackexchange.com/questions/1120978/propositional-logic-problem-sales-expenses-and-happiness-of-the-boss

        \end{itemize}

    \end{enumerate}

    \newpage

    \section{Truth Tables}
    \subsection*{Examples}

    \subsubsection*{Example 1.2.1:}

    Make a truth table for the formula $\neg (P \vee \neg Q)$.

    \begin{center}
    \begin{tabular}{ c c c c c }
        $P$ & $Q$ & $\neg Q$ & $P \vee \neg Q$ & $\neg (P \vee \neg Q)$ \\ \hline
        F & F & T & T & F \\
        F & T & F & F & T \\
        T & F & T & T & F \\
        T & T & F & T & F
    \end{tabular}
    \end{center}

    \subsubsection*{Example 1.2.2:}

    Make a truth table for the formula $\neg (P \wedge Q) \vee \neg R$.

    \begin{center}
    \begin{tabular}{ c c c c c c c }
        $P$ & $Q$ & $R$ & $\neg R$ & $P \wedge Q$ & $\neg (P \wedge Q)$ & $\neg (P \wedge Q) \vee \neg R$ \\ \hline
        F & F & F & T & F & T & T \\
        F & F & T & F & F & T & T \\
        F & T & F & T & F & T & T \\
        F & T & T & F & F & T & T \\
        T & F & F & T & F & T & T \\
        T & F & T & F & F & T & T \\
        T & T & F & T & T & F & T \\
        T & T & T & F & T & F & F
    \end{tabular}
    \end{center}

    \subsubsection*{Example 1.2.3: Determine whether the following arguments are valid.}

    Determine whether the following argument is valid.

    \begin{itemize}
        \item Let $P$ stand for the statement "John is stupid" and $Q$ for "John is lazy". Then we can represent the argument symbolically as follows:
        \begin{center}
        \begin{tabular}{ l }
            $(\neg P \wedge Q) \vee P$ \\
            $P$ \\ \hline
            $\therefore \neg Q$
        \end{tabular}
        \end{center}
    \end{itemize}

    Now we make a truth table for both premises and the conclusion.

    \begin{center}
    \begin{tabular}{ c c c c c }
            &     & \multicolumn{2}{c}{Premises}     & \multicolumn{1}{c}{Conclusion} \\
        $P$ & $Q$ & $(\neg P \wedge Q) \vee P$ & $P$ & $\neg Q$                       \\ \hline
        F   & F   & F                          & F   & T                              \\
        F   & T   & T                          & F   & F                              \\
        T   & F   & T                          & T   & T                              \\
        T   & T   & T                          & T   & F                             
    \end{tabular}
    \end{center}

    The argument is invalid since in the last line the two premises are true but the conclusion is false thus contradicting the essence of a valid argument stating that for an argument to be valid the premises cannot all be true without the conclusion being true as well.

    The other one is pretty similar... in the second example we concluded that formulas can be equilavent for example $(\neg P \wedge Q) \vee P = P \vee Q$

    \newpage

    \textbf{DeMorgan’s laws:}
    \begin{itemize}[itemsep=5pt,parsep=0pt,topsep=5pt]
        \item \begin{center}$\neg(P \land Q)$ is equivalent to $\neg P \lor \neg Q$.\end{center}
        \item \begin{center}$\neg(P \lor Q)$ is equivalent to $\neg P \land \neg Q$.\end{center}
    \end{itemize}
    
    \textbf{Commutative laws:}
    \begin{itemize}[itemsep=5pt,parsep=0pt,topsep=5pt]
        \item \begin{center}$P \land Q$ is equivalent to $Q \land P$.\end{center}
        \item \begin{center}$P \lor Q$ is equivalent to $Q \lor P$.\end{center}
    \end{itemize}
    
    \textbf{Associative laws:}
    \begin{itemize}[itemsep=5pt,parsep=0pt,topsep=5pt]
        \item \begin{center}$P \land (Q \land R)$ is equivalent to $(P \land Q) \land R$.\end{center}
        \item \begin{center}$P \lor (Q \lor R)$ is equivalent to $(P \lor Q) \lor R$.\end{center}
    \end{itemize}
    
    \textbf{Idempotent laws:}
    \begin{itemize}[itemsep=5pt,parsep=0pt,topsep=5pt]
        \item \begin{center}$P \land P$ is equivalent to $P$.\end{center}
        \item \begin{center}$P \lor P$ is equivalent to $P$.\end{center}
    \end{itemize}
    
    \textbf{Distributive laws:}
    \begin{itemize}[itemsep=5pt,parsep=0pt,topsep=5pt]
        \item \begin{center}$P \land (Q \lor R)$ is equivalent to $(P \land Q) \lor (P \land R)$.\end{center}
        \item \begin{center}$P \lor (Q \land R)$ is equivalent to $(P \lor Q) \land (P \lor R)$.\end{center}
    \end{itemize}
    
    \textbf{Absorption laws:}
    \begin{itemize}[itemsep=5pt,parsep=0pt,topsep=5pt]
        \item \begin{center}$P \lor (P \land Q)$ is equivalent to $P$.\end{center}
        \item \begin{center}$P \land (P \lor Q)$ is equivalent to $P$.\end{center}
    \end{itemize}
    
    \textbf{Double Negation law:}
    \begin{itemize}[itemsep=5pt,parsep=0pt,topsep=5pt]
        \item \begin{center}$\neg\neg P$ is equivalent to $P$.\end{center}
    \end{itemize}
    
    \subsection*{Example 1.2.5}

    \begin{align*}
        1. &\quad \neg (P \lor \neg Q) \\
        &\quad = \neg P \land \neg \neg Q \\
        &\quad = \neg P \land Q \\
        % Add more steps as needed
        2. &\quad \neg (Q \land \neg P) \lor P \\
        &\quad = \neg Q \lor \neg \neg P \lor P \\
        &\quad = \neg Q \lor P \lor P \\
        &\quad = \neg Q \lor P \\
        % Add more steps as needed
    \end{align*}

    Some equivalences are based on the fact that certain formulas are either always true or always false. For example, you can verify by making a truth table that the formula $Q \land (P \lor \neg P)$ is equivalent to just $Q$. But even before you make the truth table, you can probably see why they are equivalent. In every line of the truth table, $P \lor \neg P$ will come out true, and therefore $Q \land (P \lor \neg P)$ will come out true when $Q$ is also true, and false when $Q$ is false. Formulas that are always true, such as $P \lor \neg P$, are called tautologies. Similarly, formulas that are always false are called contradictions. For example, $P \land \neg P$ is a contradiction.

    \newpage
    \subsection*{Example 1.2.6}
    \begin{table}[htbp]
        \label{tab:logical_table}
        \begin{tabular}{cccccc}
            $P$ & $Q$ & $P \lor (Q \lor \neg P)$ & $P \land \neg(Q \lor \neg Q)$ & $P \lor \neg(Q \lor \neg Q)$ \\
            \midrule
            F   & F   & T                       & F                           & F                           \\
            F   & T   & T                       & F                           & F                           \\
            T   & F   & T                       & F                           & T                           \\
            T   & T   & T                       & F                           & T                           \\
        \end{tabular}
    \end{table}

    From the truth table we can conclude that the first statement is \emph{tautologies} since it's always true, the second is a \emph{contradiction} since it is always false and the third is neither. \\

    \textbf{Tautology laws:}
    \begin{itemize}[itemsep=5pt,parsep=0pt,topsep=5pt]
        \item \begin{center}$P \land (\text{tautology})$ is equivalent to $P$.\end{center}
        \item \begin{center}$P \lor (\text{tautology})$ is a tautology.\end{center}
        \item \begin{center}$\neg(\text{tautology})$ is a contradiction.\end{center}
    \end{itemize}

    \textbf{Contradiction laws:}
    \begin{itemize}[itemsep=5pt,parsep=0pt,topsep=5pt]
        \item \begin{center}$P \land (\text{contradiction})$ is a contradiction.\end{center}
        \item \begin{center}$P \lor (\text{contradiction})$ is equivalent to $P$.\end{center}
        \item \begin{center}$\neg(\text{contradiction})$ is a tautology.\end{center}
    \end{itemize}

    \subsection*{1.2.7}

    \begin{align*}
        1. &\quad P \lor (Q \land \neg P) \\
        &\quad = (P \lor Q) \land (P \lor \neg P) \\
        &\quad = (P \lor Q) \land \textbf{T} \\
        &\quad = P \lor Q \\
        % Add more steps as needed
        2. &\quad \neg (P \lor (Q \land \neg R)) \land Q \\
        &\quad = (\neg P \land \neg (Q \land \neg R)) \land Q \\
        &\quad = (\neg P \land (\neg Q \lor R)) \land Q \\
        &\quad = \neg P \land ((\neg Q \lor R) \land Q) \\
        &\quad = \neg P \land ((Q \land \neg Q) \lor (Q \land R)) \\
        &\quad = \neg P \land (\textbf{F} \lor (Q \land R)) \\
        &\quad = \neg P \land (Q \land R) \\
        % Add more steps as needed
    \end{align*}

    \newpage

    \subsection*{Exercises} % Section for exercises (unnumbered)
    
    \begin{enumerate}
        \item \textbf{Exercise 1:}
        \begin{itemize}
            \item[(a)] The truth table for the formula $\neg P \lor Q$,
            \begin{center}
            \begin{tabular}{cccc}
                $P$ & $Q$ & $\neg P$ & $\neg P \lor Q$ \\
                \hline
                F & F & T & T \\
                F & T & T & T \\
                T & F & F & F \\
                T & T & F & T \\
            \end{tabular}
            \end{center}
            \item[(b)] The truth table for the formula $(S \lor G) \land (\neg S \lor \neg G)$,
            \begin{center}
            \begin{tabular}{ccccccc}
                $S$ & $G$ & $\neg S$ & $\neg G$ & $S \lor G$ & $\neg S \lor \neg G$ & $(S \lor G) \land (\neg S \lor \neg G)$ \\
                \hline
                F & F & T & T & F & T & F \\
                F & T & T & F & T & T & T \\
                T & F & F & T & T & T & T \\
                T & T & F & F & T & F & F \\
            \end{tabular}
            \end{center}
        \end{itemize} 
        \item \textbf{Exercice 2:}
        \begin{itemize}
            \item[(a)] First let's simplify the formula we have,
            \begin{equation*} 
                \begin{split}
                    \neg [P \land (Q \lor \neg P)] & = \neg [(P \land Q) \lor (P \land \neg P)] \\
                    & = \neg [(P \land Q) \lor \textbf{F}] \\
                    & = \neg (P \land Q) \\
                    & = \neg P \lor \neg Q \\
                \end{split}
            \end{equation*}
            Thus the truth table of the formula $\neg [P \land (Q \lor \neg P)]$ is equivalent to the truth table of the second formula $\neg P \lor \neg Q$,

            \begin{center}
            \begin{tabular}{cccccc}
                $P$ & $Q$ & $\neg P$ & $\neg Q$ & $\neg P \lor \neg Q$ & $\neg [P \land (Q \lor \neg P)]$ \\
                \hline
                F & F & T & T & T & T \\
                F & T & T & F & T & T \\
                T & F & F & T & T & T \\
                T & T & F & F & F & F \\
            \end{tabular}
            \end{center}

            \item[(b)] The truth table of the formula $(P \lor Q) \land (\neg P \lor R)$
            \begin{center}
            \begin{tabular}{ccccccc}
                $P$ & $Q$ & $R$ & $\neg P$ & $P \lor Q$ & $\neg P \lor R$ & $(P \lor Q) \land (\neg P \lor R)$ \\
                \hline
                F & F & F & T & F & T & F \\
                F & F & T & T & F & T & F \\
                F & T & F & T & T & T & T \\
                F & T & T & T & T & T & T \\
                T & F & F & F & T & F & F \\
                T & F & T & F & T & T & T \\
                T & T & F & F & T & F & F \\
                T & T & T & F & T & T & T \\
            \end{tabular}
            \end{center}
        \end{itemize}
    
    \item \textbf{Exercice 3:}
    \begin{itemize}
        \item[(a)] The truth table of the formula $P \oplus Q$,
        \begin{center}
        \begin{tabular}{ccc}
            $P$ & $Q$ & $P \oplus Q$ \\
            \hline
            F & F & F \\
            F & T & T \\
            T & F & T \\
            T & T & F \\
        \end{tabular}
        \end{center}

        \item[(b)] The formula $P \oplus Q$ is equivalent to the formula $(P \lor Q) \land \neg (P \land Q)$ and we can prove this claim with the following truth table,
        \begin{center}
        \begin{tabular}{ccccc}
            $P$ & $Q$ & $P \lor Q$ & $(P \lor Q) \land \neg(P \land Q)$ & $P \oplus Q$ \\
            \hline
            F & F & F & F & F \\
            F & T & T & T & T \\
            T & F & T & T & T \\
            T & T & T & F & F \\
        \end{tabular}
        \end{center}

    \end{itemize}

    \item \textbf{Exercice 4:}
    \begin{itemize}
        \item[(a)] The truth table of the formula $P \downarrow Q$,
        \begin{center}
            \begin{tabular}{ccc}
                $P$ & $Q$ & $P \downarrow Q$ \\
                \hline
                F & F & T \\
                F & T & F \\
                T & F & F \\
                T & T & F \\
            \end{tabular}
        \end{center}

        \item[(b)] The formula $P \downarrow Q$ is equivalent to the formula $\neg (P \lor Q)$ and we can prove this claim with the following truth table,
        \begin{center}
            \begin{tabular}{cccc}
                $P$ & $Q$ & $\neg(P \lor Q)$ & $P \downarrow Q$ \\
                \hline
                F & F & T & T \\
                F & T & F & F \\
                T & F & F & F \\
                T & T & F & F \\
            \end{tabular}
        \end{center}

        \item[(c)] The formula $\neg P$ is equivalent to the formula $P \downarrow P$, this can be proved by the following truth table,
        \begin{center}
            \begin{tabular}{cccc}
                $P$ & $\neg P$ & $P \downarrow P$ \\
                \hline
                F & T & T \\
                F & T & T \\
                T & F & F \\
                T & F & F \\
            \end{tabular}
        \end{center} 

        The formula $P \lor Q$ is equivalent to the formula $\neg (P \downarrow Q)$, this can be proved by the following truth table,

        \begin{center}
        \begin{tabular}{ccccc}
            $P$ & $Q$ & $P \lor Q$ & $\neg (P \downarrow Q)$ \\
            \hline
            F & F & F & T \\
            F & T & T & F \\
            T & F & T & F \\
            T & T & T & F \\
        \end{tabular}
        \end{center}

        The formula $P \land Q$ is equivalent to the formula $\neg P \downarrow \neg Q$, this can be proved by the following truth table, 

        \begin{center}
        \begin{tabular}{ccccc}
            $P$ & $Q$ & $P \land Q$ & $\neg P \downarrow \neg Q$ \\
            \hline
            F & F & F & F \\
            F & T & F & F \\
            T & F & F & F \\
            T & T & T & T \\
        \end{tabular}
        \end{center}

    \end{itemize}

    \item \textbf{Exercice 6:}
    \begin{itemize}
        \item[(a)] The truth table of the formula $P \mid Q$,
        \begin{center}
        \begin{tabular}{ccc}
            $P$ & $Q$ & $P \mid Q$ \\
            \hline
            F & F & T \\
            F & T & T \\
            T & F & T \\
            T & T & F \\
        \end{tabular}
        \end{center}
        \item[(b)] The formula $P \mid Q$ is equivalent to the formula $\neg (P \land Q)$, this can be proved by the following truth table,
        \begin{center}
        \begin{tabular}{cccc}
            $P$ & $Q$ & $P \mid Q$ & $\neg (P \land Q)$ \\
            \hline
            F & F & T & T \\
            F & T & T & T \\
            T & F & T & T \\
            T & T & F & F \\
        \end{tabular}
        \end{center}
        \item[(c)] The formula $\neg P$ is equivalent to the formula $P \mid P$, this can be proved by the following truth table,
        \begin{center}
        \begin{tabular}{ccc}
            $P$ & $\neg P$ & $P \mid P$ \\
            \hline
            F & T & T \\
            T & F & F \\
        \end{tabular}
        \end{center}

        The formula $P \lor Q$ is equivalent to the formula $\neg P \mid \neg Q$, this can be proved by the following truth table,
        \begin{center}
        \begin{tabular}{cccccc}
            $P$ & $Q$ & $\neg P$ & $\neg Q$ & $P \lor Q$ & $\neg P \mid \neg Q$ \\
            \hline
            F & F & T & T & F & F \\
            F & T & T & F & T & T \\
            T & F & F & T & T & T \\
            T & T & F & F & T & T \\
        \end{tabular}
        \end{center}

        The formula $P \land Q$ is equivalent to the formula $\neg (P \mid Q)$, this can be proved by the following truth table,

        \begin{center}
        \begin{tabular}{cccc}
            $P$ & $Q$ & $P \land Q$ & $\neg (P \mid Q)$ \\
            \hline
            F & F & F & F \\
            F & T & F & F \\
            T & F & F & F \\
            T & T & T & T \\
        \end{tabular}
        \end{center}

    \end{itemize}

    \item \textbf{Exercice 8:}
    \begin{center}
    \begin{tabular}{ccccccc}
        $P$ & $Q$ & $(P \land Q) \lor (\neg P \land \neg Q)$ & $\neg P \lor Q$ & $(P \lor \neg Q) \land (Q \lor \neg P)$ & $\neg (P \lor Q)$ & $(Q \land P) \lor \neg P$ \\
        \hline
        F & F & T & T & T & T & T \\
        F & T & F & T & F & F & T \\
        T & F & F & F & F & F & F \\
        T & T & T & T & T & F & T \\
    \end{tabular}
    \end{center}

    From the following truth table, $(P \land Q) \lor (\neg P \land \neg Q)$ and $(P \lor \neg Q) \land (Q \lor \neg P)$ are equivalent formulas and we also have $\neg P \lor Q$ and $(Q \land P) \lor \neg P$ that are also equivalent.

    \item \textbf{Exercice 9:}
    \begin{center}
    \begin{tabular}{ccc}
        $P$ & $Q$ & $(P \lor Q) \land (\neg P \lor \neg Q)$ \\
        \hline
        F & F & F \\
        F & T & T \\
        T & F & T \\
        T & T & F \\
    \end{tabular}
    \end{center}

    From the above truth table, this statement is neither a tautology nor a contradiction.

    \begin{center}
    \begin{tabular}{ccc}
        $P$ & $Q$ & $(P \lor Q) \land (\neg P \land \neg Q)$ \\
        \hline
        F & F & F \\
        F & T & F \\
        T & F & F \\
        T & T & F \\
    \end{tabular}
    \end{center}

    From the above truth table, this statement is a contradiction.

    \begin{center}
    \begin{tabular}{ccc}
        $P$ & $Q$ & $(P \lor Q) \lor (\neg P \lor \neg Q)$ \\
        \hline
        F & F & T \\
        F & T & T \\
        T & F & T \\
        T & T & T \\
    \end{tabular}
    \end{center}

    From the above truth table, this statement is a tautology.

    \begin{center}
    \begin{tabular}{cccccc}
        $P$ & $Q$ & $R$ & $Q \lor \neg R$ & $P \land (Q \lor \neg R)$ & $[P \land (Q \lor \neg R)] \lor (\neg P \lor R)$ \\
        \hline
        F & F & F & T & F & T \\
        F & F & T & F & F & T \\
        F & T & F & T & F & T \\
        F & T & T & T & F & T \\
        T & F & F & T & T & T \\
        T & F & T & F & F & T \\
        T & T & F & T & T & T \\
        T & T & T & T & T & T \\
    \end{tabular}
    \end{center}

    From the above truth table, this statement is a tautology.

    \item \textbf{Exercice 10:}

    \begin{itemize}
        \item[(a)] The truth table of $\neg (P \lor Q)$ and $\neg P \land \neg Q$ is,

        \begin{center}
        \begin{tabular}{ccccc}
            $P$ & $Q$ & $\neg(P \lor Q)$ & $\neg P \land \neg Q$ \\
            \hline
            F & F & T & T \\
            F & T & F & F \\
            T & F & F & F \\
            T & T & F & F \\
        \end{tabular}
        \end{center}

        The results are equivalent for both formulas thus the formulas are equivalent, hence DeMorgan's second law has been proved to be true.

        \item[(b)] The truth table of $P \land (Q \lor R)$ and $(P \land Q) \lor (P \land R)$ is,

        \begin{center}
        \begin{tabular}{ccccc}
            $P$ & $Q$ & $R$ & $P \land (Q \lor R)$ & $(P \land Q) \lor (P \land R)$ \\
            \hline
            F & F & F & F & F \\
            F & F & T & F & F \\
            F & T & F & F & F \\
            F & T & T & F & F \\
            T & F & F & F & F \\
            T & F & T & T & T \\
            T & T & F & T & T \\
            T & T & T & T & T \\
        \end{tabular}
        \end{center}

        The truth table of $P \lor (Q \land R)$ and $(P \lor Q) \land (P \lor R)$ is,

        \begin{center}
        \begin{tabular}{ccccc}
            $P$ & $Q$ & $R$ & $P \lor (Q \land R)$ & $(P \lor Q) \land (P \lor R)$ \\
            \hline
            F & F & F & F & F \\
            F & F & T & F & F \\
            F & T & F & F & F \\
            F & T & T & F & F \\
            T & F & F & T & T \\
            T & F & T & T & T \\
            T & T & F & T & T \\
            T & T & T & T & T \\
        \end{tabular}
        \end{center}

        The results are equivalent for both formulas for both cases thus we can conclude that the distributive laws have been proved.

    \end{itemize}

    \item \textbf{Exercice 11:}
    \begin{itemize}
        \item[(a)] Simplifying the formula $\neg (\neg P \land \neg Q)$ will give us,
        \begin{equation*} 
            \begin{split}
                \neg (\neg P \land \neg Q) & = \neg \neg P \lor \neg \neg Q \\
                & = P \lor Q \\
            \end{split}
        \end{equation*}

        \item[(b)] Simplifying the formula $(P \land Q) \lor (P \land \neg Q)$ will give us,
        \begin{equation*} 
            \begin{split}
                (P \land Q) \lor (P \land \neg Q) & = P \land (Q \lor \neg Q) \\
                & = P \land \textbf{T} \\
                & = P \\
            \end{split}
        \end{equation*}

        \item[(c)] Simplifying the formula $\neg (P \land \neg Q) \lor (\neg P \land Q)$ will give us,
        \begin{equation*}
            \begin{split}
                \neg (P \land \neg Q) \lor (\neg P \land Q) & = (\neg P \lor \neg \neg Q) \lor (\neg P \land Q) \\
                & = (\neg P \lor Q) \lor (\neg P \land Q) \\
                & = Q \lor (\neg P \lor (\neg P \land Q)) \\
                & = Q \lor \neg P \\
            \end{split}
        \end{equation*}

    \end{itemize}

    \item \textbf{Exercice 12:}
    \begin{itemize}
        \item[(a)] Simplifying the formula $\neg (\neg P \lor Q) \lor (P \land \neg R)$ will give us,
        \begin{equation*}
            \begin{split}
                \neg (\neg P \lor Q) \lor (P \land \neg R) & = \neg (\neg P \lor Q) \lor (P \land \neg R) \\
                & = (\neg \neg P \land \neg Q) \lor (P \land \neg R) \\
                & = (P \land \neg Q) \lor (P \land \neg R) \\
                & = P \land (\neg Q \lor \neg R) \\
            \end{split}
        \end{equation*}
        \item[(b)] Simplifying the formula $\neg (\neg P \land Q) \lor (P \land \neg R)$ will give us,
        \begin{equation*}
            \begin{split}
                \neg (\neg P \land Q) \lor (P \land \neg R) & = (\neg \neg P \lor \neg Q) \lor (P \land \neg R) \\
                & = (P \lor \neg Q) \lor (P \land \neg R) \\
                & = \neg Q \lor (P \lor (P \land \neg R)) \\
                & = \neg Q \lor P \\
            \end{split}
        \end{equation*}
        \item[(c)] Simplifying the formula $(P \land R) \lor [\neg R \land (P \lor Q)]$ will give us,
        \begin{equation*}
            \begin{split}
                (P \land R) \lor [\neg R \land (P \lor Q)] & = (P \land R) \lor [(\neg R \land P) \lor (\neg R \land Q)] \\
                & = [(P \land R) \lor (\neg R \land P)] \lor (\neg R \land Q) \\
                & = [P \land (R \lor \neg R)] \lor (\neg R \land Q) \\
                & = (P \land \textbf{F}) \lor (\neg R \land Q) \\
                & = P \lor (\neg R \land Q) \\
            \end{split}
        \end{equation*}

    \end{itemize}

    \end{enumerate}

\end{document}
    